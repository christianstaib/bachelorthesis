\chapter{Contraction Hierarchy}

\section{Contraction Hierarchies}
\todo{background, history}

\subsection{CH idea}
This thesis explores different ways to archive the same goal: Preprocess a graph in way, that CH queries can be run. For this reason an independent defintion the CH query is required.

Consider a finite directed graph $G = (V, E)$ and a \emph{level} function $l \colon V \to \mathbb{N}$, which assignes each vertex a level, intuitively representing some information about the hierachical position of the vertex.
Let $p \colon V \times V \mapsto \mathbb{N}$ be the function that calculates \emph{the} shortest path weight and $P \colon V \times V \mapsto V^{n}$ ($n$ is not fixed, as the number of vertices visited by a shortest path is not fixed) the function that calculates the vertices visited by \emph{a} the shortest path.

With this information, the \emph{CH graph} can be defined, consisting of the \emph{upward graph} $G_{\uparrow} \coloneqq (V, E_{\uparrow})$, the \emph{downward graph} $G_{\downarrow} \coloneqq (V, E_{\downarrow})$.
$E_{\uparrow}$ is definied as follows: For each $s, t \in V$ the edge $(s, t, p(s, t))$ is in $E_{\uparrow}$ if $l(s) \leq l(t)$ and $\forall v \in P(s, t)$ (where $v \neq s$ and $v \neq t$) $l(v) < l(s)$.
Similar the $E_{\downarrow}$ is defined but all greater and greater or equal signs are flipped.

A forward search on the upward graph from vertex $s$ therefore only visits vertices with a higher level than $s$, hence the name upward graph.


\section{Vertex Contraction}
While contraction a vertex \emph{v} for each in-neighbor \emph{u} and each out-neighbor \emph{w} a shortcut \emph{(u, w)} needs to be inserted whenever \emph{(u, v, w)} is the only shortest path from \emph{u} to \emph{w}.
The correctness of the algorithm holds if more shortcuts than necessary are inserted.

Ideas how to generate shortcuts
\begin{itemize}
      \item
            witness  search

      \item
            upper bound distance oracle for \emph{(u, w)}

      \item
            all pairs shortcuts

      \item
            all pairs shortcuts with upper bound distance oracle for \emph{(u, v)} and \emph{(v, w)}.
            Only insert a shortcut if these edges are not ruled out by the distance oracle.

\end{itemize}

\section{Vertex Ordering}
The ordering of the contraction can be adaptive (TODO what does that mean) or non-adaptive.

\subsection{Adaptive vertex ordering}
Vertex ordering can be categorized in two ways: Ordering using simulated contraction and ordering not using simulated contraction.

Depending on the graph size and running time of the \emph{priority function} used (which is normally the case for simulated contraction), it might not be feasible to execute the function for all uncontracted vertices and choose the minimum.
In such cases it necessary to define ways that can combat this.

Assumption: For all vertices, the priority function is strictly increasing in the graph contraction process.

Ideas how to order adaptively:
\begin{itemize}
      \item
            Lazy popping


      \item
            Neighbor updates
\end{itemize}

\chapter{Graph contraction}
The vertices are contracted one by one (or multiple at the same time if they are \emph{independent}). The contraction order has influence to both the quality of the result (average query time, number of shortcuts added, size of hub labels) and the time needed for the contraction of the whole graph.

The order can be decided one in the beginning (\emph{fixed vertex ordering}) or after each contraction reevaluated (\emph{adaptive vertex ordering}).

simulated contraction using wittness search or heuristic

contraction can be done in the following ways:

ch:
fixed vertex ordering

adaptive node ordering
- with simulated contraction
- without simulated contraction


hl:
from ch

top down (is equivalent to ch with fixed node ordering)



\chapter{Technical Introduction}

\chapter{Method}

\chapter{Results}

\chapter{Conclusions}
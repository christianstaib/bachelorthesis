\chapter*{Kurzfassung}
Die Technik der Contraction Hierarchies (CH) und der Hierarchical Hub Labelings (HL) haben sich als äußerst wirksame Techniken zur Beschleunigung von Routenanfragen in Straßennetzwerken erwiesen. Die Anwendung auf Graphen außerhalb des Straßenverkehrs wurde weit weniger untersucht - ein Beispiel hierfür findet sich z.B. in [1] -, auch wenn in diesen Fällen ebenfalls eine schnellere Anfrage von Suchanfragen wünschenswert ist.

Ziel dieser Arbeit ist es, diese Fragestellung eingehender zu untersuchen. Es sind hierbei sowohl andere Graphtypen zu betrachten, als auch ggf. bessere Konstruktionsstrategien für CH bzw. HL zu entwerfen. Graphtypen, die potenziell von Interesse sind:

\begin{itemize}
  \item
    Sichtbarkeitsgraphen
  \item
    Gridgraphen, wie sie z.B. in Computerspielen vorkommen
  \item
    Kommunikationsgraphen
  \item
    Kollaborationsgraphen
  \item
    Linkgraphen (z.B. von Wikipedia)
\end{itemize}
